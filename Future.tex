
\section{Future Directions}


\subsection{Future direction of crowdsourcing}

A holistic crowdsourcing based data collection scheme, which includes the considerations about energy efficiency, privacy protection and incentive mechanisms, is desired for the accurate, efficient, and reliable urban informatics systems. As crowdsourcing is loose-controlled data sensing and collection paradigm, the data generated via a crowdsourcing method is nonuniformly distributed in the geographical space. We may have more data than we need at some places while we have no input at some other locations. Rather than simply assigning participants to collect data at these locations without crowdsourced data, it is more preferable to mine the spatial correlation among locations firstly and then predict the most valuable barren locations, whose data can improve the overall informatics accuracy the most, based on current available data. Such a feedback based adaptive sensing can not only determine the target incentive locations in advance to meet the timeliness of urban informatics systems, but also achieve the maximum benefit, e.g., accuracy improvement, with the minimum incentive budget. Besides, for the privacy issue in crowdsourcing, it is important to study how the same type of data can be of different privacy sensitivity in different environment and conditions. Such a differential design can not only protect the privacy of participants according to the context, but also enrich the available crowdsourcing data.

